% Options for packages loaded elsewhere
\PassOptionsToPackage{unicode}{hyperref}
\PassOptionsToPackage{hyphens}{url}
\PassOptionsToPackage{dvipsnames,svgnames,x11names}{xcolor}
%
\documentclass[runningheads]{llncs}

\usepackage{amsmath,amssymb}
\usepackage{iftex}
\ifPDFTeX
  \usepackage[T1]{fontenc}
  \usepackage[utf8]{inputenc}
  \usepackage{textcomp} % provide euro and other symbols
\else % if luatex or xetex
  \usepackage{unicode-math}
  \defaultfontfeatures{Scale=MatchLowercase}
  \defaultfontfeatures[\rmfamily]{Ligatures=TeX,Scale=1}
\fi
\usepackage{lmodern}
\ifPDFTeX\else  
    % xetex/luatex font selection
\fi
% Use upquote if available, for straight quotes in verbatim environments
\IfFileExists{upquote.sty}{\usepackage{upquote}}{}
\IfFileExists{microtype.sty}{% use microtype if available
  \usepackage[]{microtype}
  \UseMicrotypeSet[protrusion]{basicmath} % disable protrusion for tt fonts
}{}
\usepackage{xcolor}
\setlength{\emergencystretch}{3em} % prevent overfull lines
\setcounter{secnumdepth}{2}


\providecommand{\tightlist}{%
  \setlength{\itemsep}{0pt}\setlength{\parskip}{0pt}}\usepackage{longtable,booktabs,array}
\usepackage{calc} % for calculating minipage widths
% Correct order of tables after \paragraph or \subparagraph
\usepackage{etoolbox}
\makeatletter
\patchcmd\longtable{\par}{\if@noskipsec\mbox{}\fi\par}{}{}
\makeatother
% Allow footnotes in longtable head/foot
\IfFileExists{footnotehyper.sty}{\usepackage{footnotehyper}}{\usepackage{footnote}}
\makesavenoteenv{longtable}
\usepackage{graphicx}
\makeatletter
\newsavebox\pandoc@box
\newcommand*\pandocbounded[1]{% scales image to fit in text height/width
  \sbox\pandoc@box{#1}%
  \Gscale@div\@tempa{\textheight}{\dimexpr\ht\pandoc@box+\dp\pandoc@box\relax}%
  \Gscale@div\@tempb{\linewidth}{\wd\pandoc@box}%
  \ifdim\@tempb\p@<\@tempa\p@\let\@tempa\@tempb\fi% select the smaller of both
  \ifdim\@tempa\p@<\p@\scalebox{\@tempa}{\usebox\pandoc@box}%
  \else\usebox{\pandoc@box}%
  \fi%
}
% Set default figure placement to htbp
\def\fps@figure{htbp}
\makeatother

\usepackage[T1]{fontenc}
\usepackage{graphicx}
\usepackage{booktabs}
\usepackage[misc]{ifsym}
\newcommand{\corr}{(\Letter)}
\makeatletter
\@ifpackageloaded{caption}{}{\usepackage{caption}}
\AtBeginDocument{%
\ifdefined\contentsname
  \renewcommand*\contentsname{Table of contents}
\else
  \newcommand\contentsname{Table of contents}
\fi
\ifdefined\listfigurename
  \renewcommand*\listfigurename{List of Figures}
\else
  \newcommand\listfigurename{List of Figures}
\fi
\ifdefined\listtablename
  \renewcommand*\listtablename{List of Tables}
\else
  \newcommand\listtablename{List of Tables}
\fi
\ifdefined\figurename
  \renewcommand*\figurename{Fig.}
\else
  \newcommand\figurename{Fig.}
\fi
\ifdefined\tablename
  \renewcommand*\tablename{Table}
\else
  \newcommand\tablename{Table}
\fi
}
\@ifpackageloaded{float}{}{\usepackage{float}}
\floatstyle{ruled}
\@ifundefined{c@chapter}{\newfloat{codelisting}{h}{lop}}{\newfloat{codelisting}{h}{lop}[chapter]}
\floatname{codelisting}{Listing}
\newcommand*\listoflistings{\listof{codelisting}{List of Listings}}
\makeatother
\makeatletter
\makeatother
\makeatletter
\@ifpackageloaded{caption}{}{\usepackage{caption}}
\@ifpackageloaded{subcaption}{}{\usepackage{subcaption}}
\makeatother

\usepackage[numbers, compress]{natbib}
\bibliographystyle{splncs04}
\usepackage{bookmark}

\IfFileExists{xurl.sty}{\usepackage{xurl}}{} % add URL line breaks if available
\urlstyle{same} % disable monospaced font for URLs
\hypersetup{
  pdftitle={Recent Advances in Underwater Basket Weaving Under the Extreme Pressure of the Mariana Trench},
  pdfauthor={André Lauren Benjamin; Calvin Cordozar Broadus Jr.; André Patton},
  pdfkeywords={First Keyword, Second Keyword, Another Keyword.},
  colorlinks=true,
  linkcolor={blue},
  filecolor={Maroon},
  citecolor={Blue},
  urlcolor={Blue},
  pdfcreator={LaTeX via pandoc}}


\title{Recent Advances in Underwater Basket Weaving Under the Extreme
Pressure of the Mariana Trench}


\author{André Lauren Benjamin\inst{1} \and Calvin Cordozar Broadus
Jr.\inst{2,3}\corr \and André
Patton\inst{1}\orcidID{0000−1111−2222−3333}}

\authorrunning{Altmeyer et al.}

\institute{Fictional Southern University, Savannah GA
31404, USA  \email{\{a.l.benjamin,a.a.patton\}@fsu.fake} \and Fictional
West Coast University, Long Beach CA
90840, USA  \email{ccb@fwcu.fake} \and Secondary European
Affiliation, Tiergartenstr.
17, 69121 Heidelberg, Germany  \email{lncs@springer.com}}


\begin{document}
\maketitle

\begin{abstract}
The abstract should briefly summarize the contents of the paper in
150--250 words.

\keywords{First Keyword \and Second Keyword \and Another Keyword.}

\end{abstract}



\section{Introduction}\label{sec-intro}

\subsection{Main Contributions}\label{main-contributions}

Please note that the first paragraph of a section or subsection is not
indented. The first paragraph that follows a table, figure, equation
etc. does not need an indent, either.

Subsequent paragraphs, however, are indented.

\section{Related Work}\label{related-work}

\subsection{Basket Weaving}\label{basket-weaving}

\subsubsection{Underwater Basket
Weaving}\label{underwater-basket-weaving}

Only two levels of headings should be numbered. Lower level headings
remain unnumbered; they are formatted as run-in headings.

\paragraph{Underwater Basket Weaving Under Difficult
Circumstances}\label{underwater-basket-weaving-under-difficult-circumstances}

The contribution should contain no more than four levels of headings.
Table~\ref{tbl-headings} gives a summary of all heading levels.

\begin{table}

\caption{\label{tbl-headings}Table captions should be placed above the
tables.}

\centering{

\begin{tabular}{lll}
\toprule
Heading level &  Example & Font size and style\\
\midrule
Title (centered) &  {\Large\bfseries Lecture Notes} & 14 point, bold\\
1st-level heading &  {\large\bfseries 1 Introduction} & 12 point, bold\\
2nd-level heading & {\bfseries 2.1 Printing Area} & 10 point, bold\\
3rd-level heading & {\bfseries Run-in Heading in Bold.} Text follows & 10 point, bold\\
4th-level heading & {\itshape Lowest Level Heading.} Text follows & 10 point, italic\\
\bottomrule
\end{tabular}

}

\end{table}%

\section{Recent Advances from the Mariana
Trench}\label{recent-advances-from-the-mariana-trench}

Displayed equations are centered and set on a separate line.

\begin{equation}\phantomsection\label{eq-linear}{
x + y = z
}\end{equation}

Please try to avoid rasterized images for line-art diagrams and schemas.
Whenever possible, use vector graphics instead (see
Fig.~\ref{fig-duck}).

\begin{figure}[t]

\centering{

\includegraphics[width=1\linewidth,height=\textheight,keepaspectratio]{example-image-duck.pdf}

}

\caption{\label{fig-duck}A figure caption is always placed below the
illustration. Please note that short captions are centered, while long
ones are justified by the macro package automatically.}

\end{figure}%

For citations of references, we prefer the use of square brackets and
consecutive numbers. Citations using labels or the author/year
convention are also acceptable. The following bibliography provides a
sample reference list with entries for journal articles
\citep{ref_article1}, an LNCS chapter \citep{ref_lncs1}, a book
\citep{ref_book1}, proceedings without editors \citep{ref_proc1}, and a
homepage \citep{ref_url1}. Multiple citations are grouped
\citep{ref_article1, ref_lncs1, ref_book1},
\citep{ref_article1, ref_book1, ref_proc1, ref_url1}.

\section{Experiments}\label{experiments}

\subsection{Experimental Setup}\label{experimental-setup}

\subsection{Experimental Results}\label{experimental-results}

\section{Discussion}\label{discussion}

\section{Conclusion}\label{conclusion}

Of course, authors have complete freedom on how they choose to structure
their paper. Section headers from Introduction up to and including
Conclusions are completely up to the discretion of the authors; use
whichever structure you see fit. Title, Abstract, the credits
environment, and References, however, are mandatory.

\begin{credits}
\subsubsection{\ackname} A bold run-in heading in small font size at the end of the paper is
used for general acknowledgments, for example: This study was funded
by X (grant number Y).

\subsubsection{\discintname}
It is now necessary to declare any competing interests or to specifically
state that the authors have no competing interests. Please place the
statement with a bold run-in heading in small font size beneath the
(optional) acknowledgments,
for example: The authors have no competing interests to declare that are
relevant to the content of this article. Or: Author A has received research
grants from Company W. Author B has received a speaker honorarium from
Company X and owns stock in Company Y. Author C is a member of committee Z.
\end{credits}


\renewcommand\refname{References}
  \bibliography{bibliography.bib}



\end{document}
