% Options for packages loaded elsewhere
% Options for packages loaded elsewhere
\PassOptionsToPackage{unicode}{hyperref}
\PassOptionsToPackage{hyphens}{url}
\PassOptionsToPackage{dvipsnames,svgnames,x11names}{xcolor}
%
\documentclass[runningheads]{llncs}
\usepackage{xcolor}
\usepackage{amsmath,amssymb}
\setcounter{secnumdepth}{2}
\usepackage{iftex}
\ifPDFTeX
  \usepackage[T1]{fontenc}
  \usepackage[utf8]{inputenc}
  \usepackage{textcomp} % provide euro and other symbols
\else % if luatex or xetex
  \usepackage{unicode-math} % this also loads fontspec
  \defaultfontfeatures{Scale=MatchLowercase}
  \defaultfontfeatures[\rmfamily]{Ligatures=TeX,Scale=1}
\fi
\usepackage{lmodern}
\ifPDFTeX\else
  % xetex/luatex font selection
\fi
% Use upquote if available, for straight quotes in verbatim environments
\IfFileExists{upquote.sty}{\usepackage{upquote}}{}
\IfFileExists{microtype.sty}{% use microtype if available
  \usepackage[activate=false]{microtype}
  \UseMicrotypeSet[protrusion]{basicmath} % disable protrusion for tt fonts
}{}


\usepackage{longtable,booktabs,array}
\usepackage{calc} % for calculating minipage widths
% Correct order of tables after \paragraph or \subparagraph
\usepackage{etoolbox}
\makeatletter
\patchcmd\longtable{\par}{\if@noskipsec\mbox{}\fi\par}{}{}
\makeatother
% Allow footnotes in longtable head/foot
\IfFileExists{footnotehyper.sty}{\usepackage{footnotehyper}}{\usepackage{footnote}}
\makesavenoteenv{longtable}
\usepackage{graphicx}
\makeatletter
\newsavebox\pandoc@box
\newcommand*\pandocbounded[1]{% scales image to fit in text height/width
  \sbox\pandoc@box{#1}%
  \Gscale@div\@tempa{\textheight}{\dimexpr\ht\pandoc@box+\dp\pandoc@box\relax}%
  \Gscale@div\@tempb{\linewidth}{\wd\pandoc@box}%
  \ifdim\@tempb\p@<\@tempa\p@\let\@tempa\@tempb\fi% select the smaller of both
  \ifdim\@tempa\p@<\p@\scalebox{\@tempa}{\usebox\pandoc@box}%
  \else\usebox{\pandoc@box}%
  \fi%
}
% Set default figure placement to htbp
\def\fps@figure{htbp}
\makeatother





\setlength{\emergencystretch}{3em} % prevent overfull lines

\providecommand{\tightlist}{%
  \setlength{\itemsep}{0pt}\setlength{\parskip}{0pt}}



 


\usepackage[T1]{fontenc}
\usepackage{graphicx}
\usepackage{booktabs}
\usepackage[misc]{ifsym}
\newcommand{\corr}{(\Letter)}
\makeatletter

\AtBeginDocument{%
\ifdefined\contentsname
  \renewcommand*\contentsname{Table of contents}
\else
  \newcommand\contentsname{Table of contents}
\fi
\ifdefined\listfigurename
  \renewcommand*\listfigurename{List of Figures}
\else
  \newcommand\listfigurename{List of Figures}
\fi
\ifdefined\listtablename
  \renewcommand*\listtablename{List of Tables}
\else
  \newcommand\listtablename{List of Tables}
\fi
\ifdefined\figurename
  \renewcommand*\figurename{Fig.}
\else
  \newcommand\figurename{Fig.}
\fi
\ifdefined\tablename
  \renewcommand*\tablename{Table}
\else
  \newcommand\tablename{Table}
\fi
}
\@ifpackageloaded{float}{}{\usepackage{float}}
\floatstyle{ruled}
\@ifundefined{c@chapter}{\newfloat{codelisting}{h}{lop}}{\newfloat{codelisting}{h}{lop}[chapter]}
\floatname{codelisting}{Listing}
\newcommand*\listoflistings{\listof{codelisting}{List of Listings}}
\makeatother
\makeatletter
\makeatother
\makeatletter

\makeatother
\usepackage{bookmark}
\IfFileExists{xurl.sty}{\usepackage{xurl}}{} % add URL line breaks if available
\urlstyle{same}
\hypersetup{
  pdftitle={Validation challenges in large-scale tree crown segmentations from remote sensing imagery using Deep Learning: a case study in Germany},
  pdfauthor={Taimur Khan, Jasmin Krebs, Nils Nölke, Sharad Kumar Gupta, Jonathan Renkel, Caroline Arnold (order tbd)},
  pdfkeywords={First Keyword, Second Keyword, Another Keyword.},
  colorlinks=true,
  linkcolor={black},
  filecolor={Maroon},
  citecolor={Blue},
  urlcolor={Blue},
  pdfcreator={LaTeX via pandoc}}

\hypersetup{nolinks=true}

\title{Validation challenges in large-scale tree crown segmentations
from remote sensing imagery using Deep Learning: a case study in
Germany}

\titlerunning{Validation challenges in tree crown segmentations}

\author{Taimur Khan, Jasmin Krebs, Nils Nölke, Sharad Kumar Gupta,
Jonathan Renkel, Caroline Arnold (order tbd)}

\authorrunning{T. Khan et al.}

\institute{}


\begin{document}
\maketitle

\begin{abstract}
Advances in deep learning models have opened up new avenues for computer
vision tasks, like tree crown segmentations from remote sensing images.
As more and more models become available, the challenges of applying the
models to really large spatial scales (beyond the forest/plot level)
become visible. One of these challenges is the validation of the tree
crown detection and segmentation. Imprecise detections/segmentations can
lead to inaccuracies in downstream tasks in forestry and urban planning,
creating a barrier for the application of these models to real-life use
cases. Attributes of training data, e.g tree sizes, phenological stages,
number of polygon vertices, shadows, impact the prediction accuracy of
the segmentation models, hence training data should be chosen with due
diligence. Here we identify some of the key validation challenges based
on the collective experiences of not just training deep learning models,
but also applying the models to very large spatial scales in Germany
(federal state level).

\keywords{First Keyword \and Second Keyword \and Another Keyword.}

\end{abstract}



\section{Introduction}\label{introduction}

Accurately mapping tree crowns over large areas is critical for scaling
local observations to regional and global assessments of forest health,
carbon stocks, and urban greenery. In an era marked by rapid
environmental change, comprehensive mapping efforts are indispensable.
Tree crown segmentation is a cornerstone task for these objectives, as
the size and structure of a tree crown---shaped by species-specific
branching patterns, site conditions, and competition for
light---directly influences primary production. A key metric in this
context is the crown projection area (CPA), defined as the vertical
projection of the crown onto a horizontal plane. Deriving CPA provides
essential insights both at the individual tree level---where it provides
predictions of diameter, volume, and growth rates---and at the stand
level, supporting models of competition and canopy gap dynamics.
However, inaccuracies inherent in the segmentation process, particularly
when applied at large scales, can lead to biased estimations of these
critical tree variables, underscoring the urgent need to address robust
validation challenges in large-scale tree crown segmentation. While
recent advances in deep learning offer unprecedented capabilities for
automated segmentation, their robust validation at scale presents a
critical, unaddressed hurdle.

Among the various tree crown variables, crown spread presents unique
estimation challenges, both in field measurements and, critically, when
derived from remote sensing imagery. Its accurate derivation is not only
susceptible to inaccuracies in the initial segmentation process but also
relies heavily on the methods employed to extract crown spread from
segmented polygons, a process fraught with its own complexities.
Selecting an appropriate crown spread calculation method is crucial,
especially for irregularly shaped tree crown polygons. Even with
optimized calculation methods, direct matches with ground-truth data
proved low (e.g., 32\% in our experience), although a broader acceptable
margin of error (±5 m) captured 89\% of cases. These results immediately
point to difficulties in establishing precise validation metrics.
Furthermore, model performance exhibits size-dependent biases: trees
with small crowns (\textless6 m) are often overestimated, while large
crowns (\textgreater16 m) tend to be underestimated. These discrepancies
underscore the inherent limitations of deep learning segmentation models
at the extremes of the size spectrum. Critically, the quality of
validation data profoundly influences the assessment of crown spread
accuracy. For instance, our experience with the local tree inventory
(`Baumkataster' Halle/Saale), which provides only broad crown spread
ranges (5 m intervals), severely constrains the precision of validation,
highlighting the need for more granular and reliable ground-truth
datasets.

The integration of deep learning into remote sensing has opened new ways
in how we monitor and analyse ecological systems
{[}\cite{zheng2024review}, \cite{zhao2023systematic}{]}. Tree crown
segmentation, a task central to assessing forest health, carbon stocks,
and urban greenery, has especially benefited from advances in
convolutional neural networks (CNNs) and transformer-based
architectures. Approaches leveraging high-resolution imagery have
demonstrated remarkable capabilities in delineating individual tree
crowns, as evidenced by works such as \cite{weinstein2019individual},
\cite{khant2025} and \cite{freudenberg2022individual}, illustrating the
potential for scalable and automated tree mapping. The urgency of
climate change and biodiversity loss further amplifies the importance of
large-scale tree crown segmentation. Forests serve as critical carbon
sinks and biodiversity reservoirs {[}\cite{pan2024enduring}{]}, while
urban trees provide essential ecosystem services
{[}\cite{sharma2024urban}{]}. Large-scale segmentation enables
consistent, detailed monitoring of these vital resources, informing
sustainable forest management, urban planning, and policy development.
The ambition to create national and global inventories, such as those
envisioned in \cite{tolan2024very}, hinges on the reliability of such
foundational segmentation data.

However, while these deep learning models demonstrate impressive
performance in controlled environments or with limited-scale datasets,
their deployment at expansive spatial extents, such as regional or
national scales, exposes a distinct spectrum of challenges that remain
largely unaddressed. Foremost among these is validation---a critical yet
underexplored hurdle in ensuring the reliability and generalizability of
model outputs. As recent studies have shown, the performance of tree
crown segmentation models is highly sensitive to the characteristics of
the training data, such as tree size, species diversity, seasonal
variations, and image quality {[}\cite{moussaid2021tree};
\cite{cong2022citrus}{]}. These factors, when coupled with the
complexities of real-world landscapes, underscore the need for robust
and scalable validation frameworks.

This Perspective builds on our extensive experience in applying deep
learning models to large-scale tree crown segmentation across Germany,
addressing federal state-level landscapes. By synthesizing insights from
model training, deployment, and validation, we highlight the nuanced
challenges associated with assessing segmentation accuracy over
expansive and heterogeneous terrains. These challenges are not mere
technicalities---they bear significant implications for downstream
applications in forestry, conservation, and urban planning. Through this
work, we aim to not only highlight these critical validation hurdles but
also to foster a dialogue on the methodological innovations required to
ensure that deep learning advancements in remote sensing translate into
actionable insights at the scales demanded by global environmental
challenges.

\section{Case Study: Tree Crown Segmentation in Saxony and
Saxony-Anhalt,
Germany}\label{case-study-tree-crown-segmentation-in-saxony-and-saxony-anhalt-germany}

In this case study, the DeepTrees model {[}\cite{khan2025}{]} was
applied in a one-shot prediction approach using pretrained model weights
provided by DeepTrees, applied to high-resolution multispectral digital
orthophoto imagery (DOP20) covering the German federal states of Saxony
(SN) and Saxony-Anhalt (ST). Sachsen, covering approximately 18,450 km²,
and Saxony-Anhalt, spanning around 20,452 km², represent diverse
ecological and urban landscapes, ideal for assessing large-scale
segmentation model performance.

DOP20 imagery provides detailed spatial information at 20 cm resolution
per pixel, capturing visual and near infrared spectral regions essential
for accurate tree crown delineation. Using these pretrained weights, the
DeepTrees model identified approximately 218.7 million individual tree
crowns---137.3 million in Saxony and 81.4 million in Saxony-Anhalt. The
resulting segmentation dataset has been made publicly available on
Zenodo (https://zenodo.org/record/exampledoi), facilitating further
research and validation.

This comprehensive segmentation highlights substantial regional
variability in tree distribution, reflecting ecological, topographical,
and land-use gradients. The case study thus forms a practical foundation
for addressing the validation challenges detailed subsequently,
emphasizing the necessity for robust validation frameworks to ensure
reliable ecological monitoring and informed decision-making based on
large-scale automated analyses.

\emph{Figure 1: 6 panels (1 for each state). From left to right
-\textgreater{} 1) tiles with date classification 2) Land use (CORINE) +
isolines. 3) the tree segmentations.}

\section{Challenges}\label{challenges}

Validating large-scale tree crown segmentation models reveals a web of
interrelated challenges that go beyond those encountered in small,
controlled datasets. Seasonal phenology is a moving target: the same
forest can look drastically different between leaf-on summer imagery and
leaf-off winter scenes. Models trained on one season often struggle in
another, yielding inconsistent segmentation accuracy across the year.
For example, a canopy delineation that performs well on lush summer
foliage may under-segment sparse autumn crowns or miss bare branches in
winter. Such season-driven variability not only degrades model
performance but also complicates validation -- a one-shot model might
appear accurate in one season and fail in the next, raising questions
about how and when accuracy should be assessed. Incorporating
multi-season data during both training and validation is essential, as
phenological dynamics have been shown to strongly influence model
generalizability {[}\cite{moussaid2021tree}; \cite{cong2022citrus}{]}.

Spatial heterogeneity of landscapes, including terrain and illumination
differences, poses another major hurdle for both segmentation and its
validation. An algorithm that segments tree crowns flawlessly in a flat,
well-lit park may stumble in a shadow-drenched valley or on a steep
mountainside. Variations in ground elevation and slope alter the angle
of solar illumination, leading to uneven lighting and shadows that can
confuse models. In mountainous or rugged terrains, trees on north-facing
slopes might appear darker or partially occluded compared to those on
south-facing slopes with direct sun, even if they are the same species
and healthy. Such effects result in site-specific performance: models
often need fine-tuning or can suffer accuracy drops when moved to new
topographies or sensor angles. Weinstein et al.~(2020) observed this
kind of cross-site performance gap, where a tree detection model trained
in one region underperformed when applied to a different region's
imagery without adaptation, underscoring how terrain and context
influence outcomes {[}\cite{weinstein2020cross}{]}. For validation, this
means that accuracy estimates from one area may not transfer to another
-- a critical issue when assessments at national or global scales are
required.

A further fundamental challenge lies in the scarcity of accurate
ground-truth data at scale. Reliable validation hinges on high-quality
reference data (the ``ground truth''), yet collecting detailed crown
delineations over large regions is logistically difficult and expensive.
Field surveys can map individual trees with great precision (e.g.,
measuring trunks and canopy spread on the ground), but doing this over
thousands of square kilometers is infeasible. Aerial and satellite
imagery provide broader coverage for reference data, but even these
often lack the resolution or fidelity to unequivocally label each tree
crown for validation purposes. UAV (drone) campaigns can bridge the gap
by capturing very high-resolution images or LiDAR of sample areas, but
they are limited in flight range and still require extensive human
annotation to turn imagery into usable ground truth. The net result is a
mismatch of scales: our models aspire to map every tree across entire
countries, but our ground truth typically covers only small plots or
scattered samples {[}\^{}{]}. This mismatch means that validating a
``wall-to-wall'' tree map often involves extrapolating from a tiny
fraction of ground-referenced trees, introducing uncertainty. Moreover,
ground-reference datasets may not capture the full diversity of
conditions (species, canopy shapes, management regimes, etc.) present in
the larger mapping area, biasing the validation. Expanding ground-truth
collection -- through automated methods or crowdsourcing -- is thus not
just a recommendation but a necessity to overcome this validation
bottleneck (as we discuss later).

Compounding the issue of limited data is the inconsistency in reference
annotations and evaluation metrics. Even when ground-reference data
exist, their format can differ -- sometimes reference trees are marked
by a single GPS point (e.g., trunk location), sometimes by a hand-drawn
polygon outlining the crown. This creates a challenge in validation: how
do we decide if a predicted crown polygon ``matches'' a ground-truth
point, or how to handle cases where one field-mapped tree corresponds to
multiple overlapping crown segments in the image?(Figure ref)
Conversely, field crews might delineate a broad canopy as one crown
while an automated model splits it into two segments (or vice versa),
especially in dense stands where crowns merge. These ambiguities in
one-to-one correspondence make it hard to define what a ``correct''
segmentation is. Traditional pixel-wise accuracy metrics like
Intersection-over-Union (IoU) treat segmentation purely as an image
overlap problem, which may not reflect the ecological reality of
counting individual trees. IoU penalizes differences in shape or area
but doesn't account for whether the count of tree objects is correct. In
an extreme case, a model could slightly over-segment every tree
(splitting each true crown into two smaller polygons) and still achieve
a reasonable IoU, despite doubling the perceived tree count -- a
significant error for applications. On the other hand, object-centric
metrics such as panoptic segmentation quality attempt to consider both
detection and delineation of objects
{[}\cite{kirillov2019panopticsegmentation}{]}. Panoptic metrics combine
aspects of object detection (was each tree detected?) with segmentation
quality (was each crown correctly outlined?), which can be more
appropriate for tree mapping. However, even these require well-defined
ground-truth objects to compare against. When the ground truth itself is
inconsistent (e.g., how to count a clumped cluster of stems with
overlapping crowns), validation metrics struggle to fully capture model
performance. The choice of evaluation metric thus becomes non-trivial:
depending on whether one prioritizes exact crown shape, tree count, or
canopy cover, the ``best'' metric may differ. Establishing consensus on
evaluation protocols is part of the challenge -- without it, different
studies may report accuracy in incompatible ways.

There is also the issue of scale and resolution in validation reporting.
A model's accuracy can appear to vary depending on the spatial scale at
which it's evaluated. For instance, a segmentation model might achieve
high overall accuracy when averaged over an entire region, yet if one
zooms into a small test area (say a single city park or forest stand),
the error rate might be much higher or lower. This can happen if errors
are not evenly distributed: the model could perform very well in one
type of landscape (e.g., neat urban street trees) and poorly in another
(dense natural forest), and a coarse regional average could mask these
extremes. Consequently, a user working on a local conservation project
might experience worse performance than the ``headline'' accuracy
suggests, because that headline number was diluted by many easier cases
elsewhere. Ensuring that validation is robust across scales is tricky --
one must balance broad coverage with local detail. It calls for
multi-scale validation approaches, where accuracy is reported at
multiple grain sizes or stratified by landscape type. Highlighting this,
one could imagine a figure plotting model accuracy as a function of
spatial extent or across different habitat types, illustrating how
performance can drop off in specific challenging subsets despite looking
good overall. This emphasizes that heterogeneous performance is itself a
challenge to acknowledge in validation: we need methods to detect where
and why a model fails, not just an aggregate score.

To summarize, validating large-scale tree segmentation models is far
from a straightforward task. Seasonal changes, diverse terrain and
illumination, limited and inconsistent ground truth, ambiguous
evaluation criteria, and scale-dependent performance all intertwine to
create a demanding setting. These challenges are not just academic --
they directly impact how much trust we can place in AI-generated tree
maps for real-world decisions in forestry, ecology, and urban planning.
Recognizing these pain points is the first step; the next is devising
strategies to overcome them. We now turn to several recommendations
aimed at improving both models and validation practices, with an eye
toward bridging the gap between controlled experiments and the
complexity of continental-scale deployments.

\emph{Table 1: Potential validation challenges and corresponding
mitigation strategies. This table could list key challenges in
validating large-scale tree segmentations alongside proposed approaches
to address them. For example: (Challenge) Phenological variation between
seasons -- (Mitigation) use multi-temporal training data and
season-specific validation sets; (Challenge) Heterogeneous terrain and
shadowing -- (Mitigation) integrate digital elevation models and
terrain-aware modeling; (Challenge) Scarcity of ground-truth annotations
-- (Mitigation) leverage semi-automated labeling, crowdsourcing, and
active learning to expand validation data; (Challenge) Ambiguity in
crown delineation -- (Mitigation) adopt object-based accuracy metrics
and consensus protocols for what counts as a single tree; (Challenge)
Scale-dependent performance -- (Mitigation) evaluate models at multiple
spatial scales and stratify results by landscape type. This overview
would visually reinforce the narrative that each challenge has an
identifiable path forward, linking the ``Challenges'' and
``Recommendations'' sections.}

\emph{!{[}{]}{[}image1{]}}\\
\emph{Figure 2: Incomplete}

\section{Recommendations}\label{recommendations}

Addressing the above challenges requires a multi-pronged approach,
combining improvements in modeling techniques with innovations in
validation methodology and data collection. We outline here several
complementary strategies that, together, can bolster the reliability of
large-scale tree segmentation efforts. These recommendations emphasize
building more robust models through advanced training paradigms, as well
as creating better frameworks to evaluate and support those models in
real-world conditions. The overarching goal is to ensure that deep
learning advancements in tree mapping translate into trustworthy,
actionable insights at regional to global scales.

\textbf{Leverage self-supervised learning for robust representations.} A
key step towards more generalizable segmentation models is tapping into
unannotated data via self-supervised learning (SSL). Unlike traditional
supervised training which is bottlenecked by limited labeled examples,
SSL allows models to learn from the abundant pool of unlabeled remote
sensing images -- for example, by predicting missing pieces of an image
or distinguishing augmented views of the same scene. By pre-training a
model on large geospatial datasets without any human-provided labels,
the model can absorb intrinsic patterns of the landscape: textures,
shapes, seasonal changes, and other context that are common across
images. These rich foundational representations can then be fine-tuned
for tree crown segmentation with far fewer annotated samples than would
otherwise be needed. Recent efforts like PhilEO Bench demonstrate the
promise of this approach, evaluating geospatial foundation models that
were pre-trained using SSL techniques {[}\cite{fibaek2024phileo}{]}. The
results show improved performance on a range of tasks (e.g.~building
footprint extraction, road mapping) after such pre-training, compared to
models trained from scratch. In the context of trees, a model with an
SSL-pretrained backbone may already ``know'' about basic vegetation
structures, shadows, and seasonal appearances, making it more adept at
delineating crowns under varied conditions. For instance, a foundation
model trained on year-round satellite imagery might implicitly
understand the difference between a leafless oak in winter and the same
oak in summer, and thus require only a light fine-tuning to accurately
segment each. Mendieta et al.~(2023) take this further by continually
pre-training on new data distributions (a form of continual learning),
which helped build a Geospatial Foundation Model (GFM) that excelled
across multiple remote sensing tasks {[}\cite{mendieta2023gfm}{]}. Such
continual SSL training could allow segmentation models to keep improving
as more unlabeled data (e.g., new satellite images over time or from new
regions) become available, staying up-to-date with changes in landscapes
and sensor characteristics. Exploiting SSL is a powerful way to tackle
the twin issues of limited labels and dataset bias, yielding a model
that is better equipped for the diverse scenarios encountered in
large-scale mapping.

\textbf{Integrate multi-view and multi-temporal data for consistency.}
Another promising avenue is to train and evaluate models with multi-view
inputs -- that is, imagery of the same trees captured from different
angles, sensors, or times. Multi-view here can mean different things:
multi-angle (such as combining nadir and oblique aerial images),
multi-platform (satellite imagery plus drone photos), or multi-temporal
(images taken in different seasons or years). By exposing models to such
spatiotemporal diversity, we can help them learn invariances that make
segmentation more reliable. For example, a tree crown seen from directly
above might have a certain shape, but from a side angle the outline
might blend with neighbors; a model trained with both perspectives could
learn to robustly identify the crown in either view. Likewise, combining
leaf-on and leaf-off images of the same forest during training can force
a model to rely on structural features (branches, trunk hints, relative
spacing) in addition to just greenness, thereby improving its ability to
generalize across seasons. Self-supervised methods are particularly
well-suited to capitalize on multi-view data because they can be set up
to encourage the model to produce similar latent representations for
different views of the same object. Techniques such as masked
autoencoders or contrastive learning can be used on multi-view datasets
to make a model predict or align one view from another, without any
manual labels needed. Such approaches have already shown success: for
instance, researchers have used masked image modeling and contrastive
SSL on multi-view satellite imagery to boost performance on segmentation
and detection tasks, essentially teaching the model that ``these two
different-looking images actually contain the same trees''
{[}\cite{mukkavilli2023prithvi}; \cite{fibaek2024phileo}{]}. By training
with multi-view consistency, the model becomes more robust to viewpoint
and temporal changes, which directly addresses the phenology and
illumination challenges discussed earlier. Importantly, this not only
helps the model's predictions but also strengthens validation, because a
model that is consistent across views makes it easier to compare
predicted and true crowns even when the reference data comes from a
slightly different perspective or date. In practice, one could imagine a
validation scheme where the agreement of a model's output between
leaf-on and leaf-off images serves as an indicator of reliability: large
discrepancies might flag areas for further human inspection. Overall,
multi-view and multi-temporal training imbue models with a form of
contextual intelligence about the 3D and time-varying nature of trees,
making segmentation outcomes more stable across the real-world
variability that large-scale applications inevitably encompass.

\textbf{Incorporate terrain information into the modeling pipeline}. As
noted, uneven terrain can cause substantial variability in how trees
appear in images, so bringing explicit knowledge of terrain into the
segmentation process is a logical remedy. One recommendation is to fuse
digital elevation models (DEMs) or LiDAR-derived terrain data with the
imagery during model training. This could be as simple as providing the
model with an extra channel of input encoding elevation/slope, or as
complex as designing the model to separately process terrain context.
Self-supervised pre-training can be extended here too: recent work on
multisensor geospatial foundation models has shown that including
elevation data in SSL (e.g., tasking the model to distinguish imagery of
flat vs.~mountainous areas) yields better feature representations for
downstream tasks {[}\cite{han2024msGFM}{]}. By differentiating between
bare earth and above-ground structures in pre-training, the model learns
to discount illumination differences that are purely due to slope and
aspect, focusing instead on actual objects like trees. In a tree
segmentation scenario, a terrain-informed model could recognize that a
dark region in an image is a shaded hillside rather than a non-existent
gap in canopy cover, or that an elongated shape on a steep slope is
still a single tree crown albeit skewed by perspective. Incorporating
terrain data directly addresses the challenge of spatial heterogeneity:
it provides a reference frame to normalize out some variability. This
can improve validation as well -- for example, error analysis can be
stratified by terrain class to ensure a model works not just on average,
but on hilltops and valleys alike. We recommend that future segmentation
models, especially for large regions with varied topography, adopt
terrain-aware training strategies. Even if a full DEM is not available
everywhere, approximating slope from the imagery or using coarse global
elevation data could still offer benefits. Ultimately, bridging the gap
between pixel appearance and real-world topography makes the model's
understanding more physical and generalizable, reducing surprises when
it's deployed on a new landscape.

\textbf{Focus model attention on domain-specific features.} Beyond data
augmentation and multi-source inputs, improvements in the model
architecture and training objectives themselves can yield more reliable
segmentation. One intriguing direction is the use of feature-guided
masked autoencoders or similar techniques that encourage the model to
learn high-level features rather than getting bogged down in pixel-level
noise. In remote sensing, not all pixels are equal -- the spectral
signature of a healthy tree canopy, for instance, is characterized by
certain reflectance patterns (like high near-infrared reflectance for
foliage), and textural cues can differentiate a tree from grass. A
vanilla model might or might not latch onto these subtle cues. However,
a feature-guided approach explicitly trains the model to reconstruct or
predict meaningful feature representations (such as vegetation indices
or edge maps) instead of raw pixel intensity. For example, a recent
approach called FG-MAE (Feature Guided Masked Autoencoder) masks out
parts of an image and tasks the model with predicting domain-relevant
features (like a NDVI -- Normalized Difference Vegetation Index image,
or other engineered representations) for the masked region rather than
the raw pixels. This forces the model to infer what type of object
should be there, not just to copy textures, thereby learning a more
semantic understanding of the scene {[}\cite{mukkavilli2023prithvi}{]}.
Applying this idea to tree segmentation, we could pre-train models to
predict features that highlight vegetation structure (perhaps oriented
gradients that capture tree crown edges, or canopy height estimations
from stereo images) so that the model's internal representations become
highly attuned to ``tree-ness.'' When fine-tuned for segmentation, such
a model may be better at delineating tangled canopies or differentiating
trees from confusing background elements, because it has learned to
focus on the attributes that define a tree in imagery. Early studies in
multispectral and SAR domains have found that this approach yields
improved performance in complex environments
{[}\cite{mukkavilli2023prithvi}{]}. We recommend integrating
feature-guidance in training for large-scale tree mapping, especially in
areas with complex backgrounds (e.g., urban environments where trees
mingle with buildings). Not only does this likely boost accuracy, it
could also produce more interpretable model outputs or uncertainties --
a model that knows what features it's looking for might provide
reasoning (explicit or implicit) for its segmentation, which in turn
aids validation and error diagnosis.

\textbf{Expand validation beyond pixel agreement -- use ecological
consistency checks.} Traditional validation of segmentation focuses on
geometric overlap with ground truth shapes, but for tree mapping we can
also exploit well-established ecological relationships as an additional
form of validation. Trees have allometric relationships -- mathematical
links between dimensions like trunk diameter, height, crown diameter,
and biomass. These relationships have been measured in forestry for
decades. For example, a tree of a given height typically has a crown of
roughly proportional diameter, and there are known bounds on how big a
crown can get for a given trunk size. We propose using such allometric
equations as a sanity check for segmentation outputs. If a model's
predicted tree crowns violate basic allometry (say, a tiny tree height
but a huge crown width, or a cluster of crowns whose total basal area
implies an impossible biomass for that area), it might indicate errors
in the segmentation or missing trees. One could quantitatively compare
the distribution of predicted crown sizes and tree heights (if height
data or estimates are available) against expected distributions from
field data. Significant deviations could flag problems: for instance, if
the model frequently delineates very large crowns that in reality would
correspond to 80-meter tall trees (which don't exist in the region),
those are likely over-segmentation artifacts. Conversely, if predicted
crowns are all very small in an old-growth forest where trees should be
large, the model might be under-segmenting (splitting one crown into
many). Researchers have indeed used this kind of approach in related
contexts; for example, Song et al.~(2023) employed Gaussian process
regression on tree height to estimate biomass, demonstrating how linking
remote sensing outputs to allometric models can validate whether the
outputs make ecological sense {[}\cite{song2023biomass}{]}. In practice,
implementing allometric checks means bringing in additional data or
models (e.g., a LiDAR-derived tree height map, or species-specific
allometric formulas from forestry literature) and cross-verifying the
plausibility of the AI-generated tree map. This recommendation shifts
validation from a purely computer-vision perspective to an
application-oriented perspective: after all, if the ultimate goal is to
use these maps for carbon accounting, biodiversity, or forestry, then
passing an ecological reality check is as important as scoring well on
IoU.

\textbf{Establish community benchmarks and evaluation frameworks.} The
field of geospatial AI is recognizing the value of standard benchmarks
-- datasets and metrics on which different methods can be compared in a
reproducible way. For computer vision in general, ImageNet and COCO
served this role, and now geospatial analogues are emerging. Efforts
like GEO-Bench and PANGAEA have started assembling diverse geospatial
tasks into comprehensive evaluation suites, underscoring the importance
of benchmarking models across varying conditions
{[}\cite{marsocci2024pangaea}{]}. We recommend developing similar
benchmark datasets and challenges specifically for tree crown
segmentation and mapping. These benchmarks should cover a wide range of
landscapes (different forest types, urban trees, plantations, etc.),
seasons, and remote sensing data sources, reflecting the real-world
diversity highlighted in our Challenges section. Importantly, a common
benchmark would promote consensus on validation metrics -- if the
community agrees to evaluate on, say, a mix of IoU for segmentation
quality and perhaps a detection metric for counting accuracy, it would
standardize how results are reported. This comparability accelerates
progress: researchers can pinpoint which innovations truly lead to
better performance and robustness. Moreover, a shared evaluation
framework can include protocols for multi-scale assessment (e.g.,
requiring methods to report not just overall accuracy but also
performance on hard subsets like ``dense forest canopy'' vs ``isolated
trees'') and encourage inclusion of auxiliary checks like the allometric
consistency mentioned above. In short, a benchmarking initiative for
large-scale tree segmentation would provide an invaluable feedback loop,
where challenges discovered by one team (e.g., failure cases in a
certain region or condition) become part of the test set that everyone
then tries to solve. This kind of iterative improvement cycle has been
crucial in other domains of AI. We envision something like an annual
challenge where models compete on a standard large-scale tree mapping
task -- pushing methods to be not only accurate but also general and
validation-friendly. By establishing these community standards, we move
toward a future where claims of model performance are transparent and
believable, because they've been vetted on a broad, agreed-upon spectrum
of scenarios.

\textbf{Innovate in ground-truth data collection and labeling.} Lastly,
none of the above model improvements obviates the need for better and
more plentiful validation data. We therefore recommend parallel efforts
to enhance ground-truth collection through automation and crowdsourcing.
On the automation side, advances in high-resolution imaging and onboard
AI mean that drones or small aircraft could be deployed to automatically
detect and delineate trees in sample areas, providing semi-automated
annotations that experts only need to lightly verify. For instance, a
drone flying over a forest could use its own simpler model to suggest
crown boundaries, which are then corrected by a human specialist --
vastly speeding up the annotation process compared to drawing polygons
from scratch. Similarly, leveraging crowdsourcing and citizen science
could dramatically expand validation datasets: non-experts can be asked
to label trees on accessible platforms (especially for easy tasks like
clicking the center of a visible tree), and with enough redundancy and
quality control, these contributions can rival expert labels. Projects
for mapping trees in cities (e.g., through apps that engage the public
to identify street trees) hint at this potential. Of course, when using
crowd data, one must account for variability in skill and ensure
rigorous cleaning and validation of the contributed labels -- but as a
complementary source of truth data, it could fill gaps in areas
professionals have not covered. Another approach is active learning,
where the model itself guides where more data is needed: the model can
flag areas or examples where it is most uncertain or makes conflicting
predictions (say, an area of weirdly segmented crowns), and those areas
would be prioritized for human labeling. This way, instead of randomly
sampling locations for ground truth, we focus effort on the most
informative samples -- those likely to teach the model something new or
reveal a blind spot. Active learning strategies have been shown to
greatly reduce the amount of data needed for training in other remote
sensing tasks by smartly selecting the right samples to label
{[}\^{}{]}. Integrating an active learning loop into large-scale
validation means we continually refine the ground truth in the areas
that matter most, thereby improving the model in a targeted fashion.
Finally, we should not overlook the role of policy and open data
initiatives: governments and organizations conducting tree inventories
or LiDAR scans should be encouraged to share these as open benchmarks.
Even if such datasets are not perfectly aligned with satellite imagery,
they can often be matched or used to validate parts of a map. In
summary, closing the validation data gap will require creativity --
using machines to help label, using the crowd to scale up, and using the
model's own intelligence to guide where to look next. These efforts
complement the technical recommendations above: better models make use
of the new data more effectively, and better data enables more powerful
models.

By pursuing these recommendations in concert, the field can make
significant strides toward accurate and trustworthy large-scale tree
mapping. What we envision is an end-to-end pipeline where robust models
(forged through self-supervised, multi-view, terrain-aware training)
produce segmentation maps that are continuously checked against both
classical metrics and real-world plausibility (via benchmarks and
allometric/field validations), and where the feedback from validation
drives further improvement of the models through active learning and
expanded training data.


\bibliographystyle{splncs04}
\bibliography{bibliography}

\end{document}
